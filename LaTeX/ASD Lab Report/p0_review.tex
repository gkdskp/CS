\documentclass{article}
\usepackage[utf8]{inputenc}
\usepackage{url}
\usepackage{graphicx}
\usepackage{geometry}
\geometry{a4paper, left=20mm, right=20mm, top=20mm, bottom=20mm}
 \usepackage{float}
 \documentclass[13pt,oneside]{book}
\usepackage[margin=1.2in]{geometry}
\usepackage[toc,page]{appendix}
\usepackage{graphicx}
\usepackage{natbib}
\usepackage{lipsum}
\usepackage{caption}

\begin{document}

\captionsetup[figure]{margin=1.5cm,font=small,labelfont={bf},name={Figure},labelsep=colon,textfont={it}}
\captionsetup[table]{margin=1.5cm,font=small,labelfont={bf},name={Table},labelsep=colon,textfont={it}}
\setlipsumdefault{1}

\begin{titlepage}
\begin{center}
{\LARGE College Of Engineering Trivandrum}\\[3cm]
\linespread{1.2}\huge {\bfseries Application Software Development Lab}\\[3cm]
\linespread{1}
\includegraphics[width=5cm]{img/emblem.jpeg}\\[3cm]
{\Large Gokul K\\ S5  CSE Roll No:21\\ TVE18CS021 }\\[1cm]


\textit{ }\\[2cm]
Department of Computer Science\\[0.2cm]
\today
\end{center}

\end{titlepage}

\newpage

\begin{frame}{}
    \centering
    \hspace*{-0.5cm}
    $\vcenter{\hbox{\includegraphics[width=1.5cm]{img/index.jpeg}}}$
    $\vcenter{\resizebox{0.95\textwidth}{!}{
        \begin{tabular}{c}
             CS333 - Application Software Development Lab $\cdot$ 2020 $\cdot$   \\
             \hline 
        \end{tabular}
    }}$
\end{frame}
\section*{Cycle 1}
\section*{Expt 0}
\begin{center}
    \Large{Review of SQL}
\end{center}

\section{Aim}
\large{To review the commands learned in database theory}

\section{Introduction}
\large Structure Query Language(SQL) is a database query language used for storing and managing data in Relational DBMS. It is used to create, maintain and retrieve the data from relational databases like MySQL, Oracle, SQL Server, PostGre, etc. The recent ISO standard version of SQL is SQL:2019.
\section{Commands}
\subsubsection{Data Definition Language}
\begin{itemize}
    \item {CREATE}\newline
    To create a table in a database\newline
    Syntax: 
    \begin{verbatim}
        CREATE TABLE table_name (
            field1 TYPE1 ..,
            field2 TYPE2
        );
    \end{verbatim}
    
    
    \item ALTER
    \newline To alter an existing table
    \newline Syntax
    \begin{verbatim}
        ALTER TABLE table_name
        ADD column_name datatype
        DROP column_name
        MODIFY column_name data_type;
    \end{verbatim}
    
    \item {DROP}\newline
    To delete a table in a database\newline
    Syntax: 
    \begin{verbatim}
        DROP TABLE table_name;
    \end{verbatim}
\end{itemize}

\subsubsection{Data Manipulation Language}
    \begin{itemize}
        \item {SELECT}\newline
        To select and filter rows from tables in a database\newline
        Syntax: 
        \begin{verbatim}
            SELECT column1, column2, ...
            FROM table_name, table_name2
            WHERE condition;
        \end{verbatim}
    
    
        \item INSERT
        \newline Insert new rows into table
        \newline Syntax
        \begin{verbatim}
            INSERT INTO table_name (column1, column2, column3, ...)
            VALUES (value1, value2, value3, ...);
        \end{verbatim}
 
        \item {UPDATE}\newline
        To update existing rows in a table in database\newline
        Syntax: 
        \begin{verbatim}
            UPDATE table_name
            SET column1 = value1, column2 = value2, ...
            WHERE condition;
        \end{verbatim}
        
        \item DELETE
        \newline Delete records(rows) from the table
        \newline Syntax:
        \begin{verbatim}
            DELETE FROM table_name WHERE condition;
        \end{verbatim}
    \end{itemize}
    
 
\end{document}