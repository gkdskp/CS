\documentclass{article}
\usepackage[T1]{fontenc}

\usepackage{graphicx}
\usepackage{listings}
\begin{document}

\title{FOSS Lab Exam Report}
\author{Gokul K\\[2\baselineskip]
Roll Number: 21 ( TVE18CS021 )\\[2\baselineskip]}
\date{24 July 2020}

\maketitle
\newpage
\section{Problem 1}
\subsection{Problem Statement}
If p is the perimeter of a right angle triangle with integral length sides, {a,b,c}, there are exactly three solutions for p = 120.  {20,48,52}, {24,45,51}, {30,40,50} 
For which value of p ≤ 1000, is the number of solutions maximised? Write a shell script   to find the same.

\subsection{Theory}
A right angle triangle with sides a, b, c satisifies Pythagoras theorem
\(a^2 + b^2 = c^2\)

Let the given perimeter be P. Then
\(a + b + c = P\)


So we will find all c that satisifies both condition

\subsection{Working}
In this program we will find the number of solution for each perimeter from 1 to 1000. For each number we will loop through i from 0 to 1000, j from 1 to 1000 and find c such that it holds both pythagoras theorem and the perimeter.
If such c is found a counter is incremented. The final value of the counter for each number is redirected to a file which is then sorted and only the first line is printed

\subsection{Files}
To run the file, extract the source zip file
and execute
\begin{verbatim}
    sh Gokul_TVE18CS021_program1.sh
\end{verbatim}

\subsection{Test case}
Since the output is constant only one test run is conducted which outputted:
\begin{verbatim}
    The number with maximum number of solution and its number of solution is 
    840 8
\end{verbatim}


\subsection{Source Code}
\begin{verbatim}
    #! /bin/bash

# If p is the perimeter of a right angle triangle with integral 
# length sides, {a,b,c}, there are exactly three solutions for p = 120.  
# {20,48,52}, {24,45,51}, {30,40,50} 
# For which value of p ≤ 1000, is the number of solutions maximised? 
# Write a shell script   to find the same.

# Since the runtime of the program is O(P^2)
# the program takes quite a while when P is as large as
# 1000

NoOfSolutions() {
        if (($1 % 2 != 0)); then
                echo "0"
        else
                count=0

                for ((i = 1; i <= $1; i++)); do
                        for ((j = $((i + 1)); j <= $1; j++)); do
                                c=$(($1 - i - j))
                                if ((i * i + j * j == c * c)); then
                                        count=$((count + 1))
                                fi
                        done
                done
                echo $count
        fi
}

for ((i = 1; i <= 1000; i++)); do
        echo "$i $(NoOfSolutions $i)" >>temp
done

echo "The number with maximum number of solution and its number of solution is is"
echo `cat temp | sort -k 2 -r | head -n 1`

}
\end{verbatim}

\newpage
\section{Problem 2}
\subsection{Problem Statement}
Write a script to count the line of a file, to also compute the total number of bytes and the total number of words in the input files.

\subsection{Theory}
wc is a UNIX command used to count the number of lines, characters, words, size etc of a file

\subsection{Working}
In this program we will use wc command to find the asked properties of the file.

\subsection{Files}
To run the file, extract the source zip file
and execute
\begin{verbatim}
    sh Gokul_TVE18CS021_program2.sh /path/to/file
\end{verbatim}

\subsection{Test case}
On the Gokul\_TVE18CS021\_program2.sh itself
\begin{verbatim}
No of lines: 21
No of bytes: 469
No of words: 97
\end{verbatim}

On the Gokul\_TVE18CS021\_program1.sh
\begin{verbatim}
No of lines: 37
No of bytes: 902
No of words: 182
\end{verbatim}

On the exam report
\begin{verbatim}
No of lines: 710
No of bytes: 83765
No of words: 2684
\end{verbatim}

\subsection{Source Code}
\begin{verbatim}
#! /bin/bash

# Write a script to count the line of a file, to 
# also compute the total number of bytes and the 
# total number of words in the input files.

if [[ $# != 1 ]]
then
        echo "Give a file as an argument"
fi

if [[ !(-f $1) ]]
then
        echo "$1 is not a valid file"
fi

noOfLines=$( cat $1 | wc -l )
noOfBytes=$( cat $1 | wc -c )
noOfWords=$( cat $1 | wc -w )

printf "No of lines: %s\nNo of bytes: %s\nNo of words: %s\n" \
"$noOfLines" \
"$noOfBytes" \
"$noOfWords"

\end{verbatim}

\end{document}

