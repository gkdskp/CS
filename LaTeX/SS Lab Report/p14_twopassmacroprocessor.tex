\documentclass[13pt,oneside]{book}
\usepackage[utf8]{inputenc}
\usepackage{url}
\usepackage{listings}
\usepackage{graphicx}

\usepackage{geometry}
\geometry{a4paper, left=20mm, right=20mm, top=20mm, bottom=20mm}
\usepackage[margin=1.2in]{geometry}
\usepackage[toc,page]{appendix}
\usepackage{graphicx}
\usepackage{natbib}
\usepackage{lipsum}
\usepackage{caption}

\begin{document}

\captionsetup[figure]{margin=1.5cm,font=small,labelfont={bf},name={Figure},labelsep=colon,textfont={it}}
\captionsetup[table]{margin=1.5cm,font=small,labelfont={bf},name={Table},labelsep=colon,textfont={it}}
\setlipsumdefault{1}

\begin{titlepage}
\begin{center}
{\LARGE College Of Engineering Trivandrum}\\[3cm]
\linespread{1.2}\huge {\bfseries System Software Lab}\\[3cm]
\linespread{1}
\includegraphics[width=5cm]{img/emblem.jpeg}\\[3cm]
{\Large GOKUL K\\ S5  CSE \\ Roll No:21\\ TVE18CS021 }\\[1cm]


\textit{ }\\[2cm]
Department of Computer Science\\[0.2cm]
\today
\end{center}

\end{titlepage}

\newpage

\begin{frame}{}
    \centering
    \hspace*{-0.5cm}
    $\vcenter{\hbox{\includegraphics[width=1.5cm]{img/emblem.jpeg}}}$
    $\vcenter{\resizebox{0.95\textwidth}{!}{
        \begin{tabular}{c}
             CS331 - System Software Lab $\cdot$ 2020 $\cdot$   \\
             \hline 
        \end{tabular}
    }}$
\end{frame}
\section*{Cycle 2}
\section*{Expt 14}
\begin{center}
    \Large{Implementation of Two Pass Macro Processor}
\end{center}
\section*{Aim}
\large
To implement a two pass macro processor

\section*{Algorithm} 
    \begin{verbatim}
Pass-one
1. Scan the input program for MACRO mnemonic
2. If found, add its label to NAMTAB
3. Do until MEND mnemonic is found,
	3.1           child: Text("Feed empty. Pull to refresh"));
	else if (index == 0)
	  return Column(
		mainAxisSize: MainAxisSize.min,
		children: [
		  Text(
			"Pull to refresh",
			style: Theme.of(context)
				.textTheme
				.caption
				.copyWith(fontSize: 10.0),
		  ),
		  buildTitleWidget(),
		],
  Copy the opcode and operand to DEFTAB
	\end{verbatim}

Pass-two
\begin{verbatim}
1. Scan the input program for the macro name mnemonic found in NAMTAB
2. If found, Print the contents of DEFTAB replacing arguments by operands in 1
3. Print the replaced operands to output
4. Else print the label opcode operand to output except for the macro definition
\end{verbatim}

\section*{Source Code}
\small

main.c
\begin{lstlisting}[language=C]
#include <stdio.h>
#include <stdlib.h>
#include <string.h>

void cat(FILE * fp) {
	char line[50];
	rewind(fp);
	while(! feof(fp)) {
		fscanf(fp, "%[^\n]\n", line);
		printf("%s\n", line);
	}
}

void pass_one() {
  FILE *input, *namtab, *deftab;
  char label[20], opcode[20], operand[20];

  if (!(input = fopen("input.txt", "r"))) {
	printf("Please make sure input.txt exists\n");
	exit(0);
  }

  namtab = fopen("namtab.txt", "w+");
  deftab = fopen("deftab.txt", "w+");

  do {
	fscanf(input, "%s %s %s", label, opcode, operand);

	if (strcmp(opcode, "MACRO") == 0) {
	  fprintf(namtab, "%s\n", label);
	  fprintf(deftab, "%s %s\n", label, operand);
	}

	else
	  fprintf(deftab, "%s %s\n", opcode, operand);
	  
  } while (strcmp(opcode, "MEND"));

  printf("Pass one completed\n");
  
  printf("Output of namtab.txt: \n");
  cat(namtab);
  printf("Output of deftab.txt: \n");
  cat(deftab);

  fclose(input);
  fclose(namtab);
  fclose(deftab);
}

void pass_two() {
  FILE *input, *deftab, *namtab, *argtab, *output;
  char label[20], opcode[20], operand[20], macro_name[20], *token;
  char deftab_opcode[20], deftab_operand[20], macro_args[20][20],
	  operand_sub[20];
  size_t macro_args_idx = 0;

  if (!(input = fopen("input.txt", "r")) ||
	  !(deftab = fopen("deftab.txt", "r")) ||
	  !(namtab = fopen("namtab.txt", "r"))) {
	printf("Pass two failed - Verify the output of pass one\n");
  }

  argtab = fopen("argtab.txt", "w+");
  output = fopen("output.txt", "w+");

  fscanf(namtab, "%s", macro_name);

  do {
	fscanf(input, "%s %s %s", label, opcode, operand);

	if (strcmp(opcode, "MACRO") == 0) {
	  token = strtok(operand, ",");
	  while (token != NULL) {
		strcpy(macro_args[macro_args_idx++], token);
		token = strtok(NULL, ",");
	  }

	  while (strcmp(opcode, "MEND") != 0)
		fscanf(input, "%s %s %s", label, opcode, operand);
	}

	token = NULL;

	if (strcmp(opcode, macro_name) == 0) {
	  token = strtok(operand, ",");
	  while (token != NULL) {
		fprintf(argtab, "%s\n", token);
		token = strtok(NULL, ",");
	  }

	  token = NULL;

	  fscanf(deftab, "%s %s", deftab_opcode, deftab_operand);
	  while (strcmp(deftab_opcode, "MEND") != 0) {
		fprintf(output, "- %s ", deftab_opcode);
		token = strtok(deftab_operand, ",");
		while (token != NULL) {
		  if (token[0] == '&') {
			rewind(argtab);
			macro_args_idx = 0;
			while (strcmp(macro_args[macro_args_idx++], token)) {
			  fscanf(argtab, "%s", operand_sub);
			}
			fscanf(argtab, "%s", operand_sub);
			fprintf(output, "%s,", operand_sub);
		  } else {
			  fprintf(output, "%s,", token);
		  }
		  token = strtok(NULL, ",");
		}
		fseek(output, -1, SEEK_CUR); // To remove the trailing comma
		fprintf(output, "\n");
		fscanf(deftab, "%s %s", deftab_opcode, deftab_operand);
	  }
	} else {
		if(strcmp(opcode, "MEND") != 0)
		fprintf(output, "%s %s %s\n", label, opcode, operand);
	}

  } while (strcmp(opcode, "END"));

  printf("\nPass two completed\n");

  printf("Output of argtab.txt");
  cat(argtab);
  printf("Output of output.txt\n");
  cat(output);

  fclose(argtab);
  fclose(deftab);
  fclose(namtab);
  fclose(input);
  fclose(output);
}

void main() {
  pass_one();
  pass_two();
}
\end{lstlisting}
input.txt
\begin{verbatim}
EX1	MACRO &A,&B
-	LDA	&A
-	STA	&B
-	MEND	-
SAMPLE	START	1000
-	EX1	N1,N2
N1	RESW	1
N2	RESW	1
-	END	-	
	\end{verbatim}
	
    \section*{Output}
    \includegraphics[width=\textwidth]{img/p14.png}
     
\Large
\section*{Result}
\large
Two pass macro processor is implemented and its output is verified
\end{document}